%% using aastex version 6.3
\documentclass[twocolumn]{aastex63}

%%%%%%%%%%%%%%%%%%%%%%%%%%%%%%%%%%%%%%%%%%%%%%%%%%%%%%%%%%%%%%%%%%%%%%%%%%%%%%%%
%%
%% The following section defines new commands for comments from co-authors
%%
\definecolor{DarkOrange}{RGB}{204, 85, 0}
\definecolor{LincolnGreen}{RGB}{17, 102, 0}
\definecolor{Rust}{HTML}{9B4F0F}
\definecolor{DarkCyan}{HTML}{008B8B}
\definecolor{MediumAquaMarine}{HTML}{66CDAA}


\def\ion#1#2{#1$\;${\footnotesize\rm{#2}}\relax}

\newcommand{\xander}[1]{{\color{red} KM: \textbf{#1}}}
\newcommand{\aam}[1]{{\color{DarkOrange} aam: \textbf{#1}}}

\newcommand{\todo}[1]{{\color{magenta} to-do: {#1}}}

\usepackage{lineno}
% \linenumbers
\usepackage{pifont}% http://ctan.org/pkg/pifont
\newcommand{\cmark}{\ding{51}}%
\newcommand{\xmark}{\ding{55}}%

%%
%%%%%%%%%%%%%%%%%%%%%%%%%%%%%%%%%%%%%%%%%%%%%%%%%%%%%%%%%%%%%%%%%%%%%%%%%%%%%%%%


%% Reintroduced the \received and \accepted commands from AASTeX v5.2
\received{\today}
\revised{}
\accepted{}
%% Command to document which AAS Journal the manuscript was submitted to.
%% Adds "Submitted to " the argument.
\submitjournal{}

%% For manuscript that include authors in collaborations, AASTeX v6.3
%% builds on the \collaboration command to allow greater freedom to 
%% keep the traditional author+affiliation information but only show
%% subsets. The \collaboration command now must appear AFTER the group
%% of authors in the collaboration and it takes TWO arguments. The last
%% is still the collaboration identifier. The text given in this
%% argument is what will be shown in the manuscript. The first argument
%% is the number of author above the \collaboration command to show with
%% the collaboration text. If there are authors that are not part of any
%% collaboration the \nocollaboration command is used. This command takes
%% one argument which is also the number of authors above to show. A
%% dashed line is shown to indicate no collaboration. This example manuscript
%% shows how these commands work to display specific set of authors 
%% on the front page.
%%
%% For manuscript without any need to use \collaboration the 
%% \AuthorCollaborationLimit command from v6.2 can still be used to 
%% show a subset of authors.
%
%\AuthorCollaborationLimit=2
%
%% will only show Schwarz & Muench on the front page of the manuscript
%% (assuming the \collaboration and \nocollaboration commands are
%% commented out).
%%
%% Note that all of the author will be shown in the published article.
%% This feature is meant to be used prior to acceptance to make the
%% front end of a long author article more manageable. Please do not use
%% this functionality for manuscripts with less than 20 authors. Conversely,
%% please do use this when the number of authors exceeds 40.
%%
%% Use \allauthors at the manuscript end to show the full author list.
%% This command should only be used with \AuthorCollaborationLimit is used.

%% The following command can be used to set the latex table counters.  It
%% is needed in this document because it uses a mix of latex tabular and
%% AASTeX deluxetables.  In general it should not be needed.
%\setcounter{table}{1}

%%%%%%%%%%%%%%%%%%%%%%%%%%%%%%%%%%%%%%%%%%%%%%%%%%%%%%%%%%%%%%%%%%%%%%%%%%%%%%%%
%%
%% The following section outlines numerous optional output that
%% can be displayed in the front matter or as running meta-data.
%%
%% If you wish, you may supply running head information, although
%% this information may be modified by the editorial offices.
\shorttitle{}
\shortauthors{}
%%
%% You can add a light gray and diagonal water-mark to the first page 
%% with this command:
%% \watermark{text}
%% where "text", e.g. DRAFT, is the text to appear.  If the text is 
%% long you can control the water-mark size with:
%% \setwatermarkfontsize{dimension}
%% where dimension is any recognized LaTeX dimension, e.g. pt, in, etc.
%%
%%%%%%%%%%%%%%%%%%%%%%%%%%%%%%%%%%%%%%%%%%%%%%%%%%%%%%%%%%%%%%%%%%%%%%%%%%%%%%%%
\graphicspath{{./}{figures/}}
%% This is the end of the preamble.  Indicate the beginning of the
%% manuscript itself with \begin{document}.

\begin{document}

\title{A Morphological Classification Model to Identify Unresolved PanSTARRS1 Sources II: Update to the PS1 Point Source Catalog}

%% LaTeX will automatically break titles if they run longer than
%% one line. However, you may use \\ to force a line break if
%% you desire. In v6.3 you can include a footnote in the title.

%% A significant change from earlier AASTEX versions is in the structure for 
%% calling author and affiliations. The change was necessary to implement 
%% auto-indexing of affiliations which prior was a manual process that could 
%% easily be tedious in large author manuscripts.
%%
%% The \author command is the same as before except it now takes an optional
%% argument which is the 16 digit ORCID. The syntax is:
%% \author[xxxx-xxxx-xxxx-xxxx]{Author Name}
%%
%% This will hyperlink the author name to the author's ORCID page. Note that
%% during compilation, LaTeX will do some limited checking of the format of
%% the ID to make sure it is valid. If the "orcid-ID.png" image file is 
%% present or in the LaTeX pathway, the OrcID icon will appear next to
%% the authors name.
%%
%% Use \affiliation for affiliation information. The old \affil is now aliased
%% to \affiliation. AASTeX v6.3 will automatically index these in the header.
%% When a duplicate is found its index will be the same as its previous entry.
%%
%% Note that \altaffilmark and \altaffiltext have been removed and thus 
%% can not be used to document secondary affiliations. If they are used latex
%% will issue a specific error message and quit. Please use multiple 
%% \affiliation calls for to document more than one affiliation.
%%
%% The new \altaffiliation can be used to indicate some secondary information
%% such as fellowships. This command produces a non-numeric footnote that is
%% set away from the numeric \affiliation footnotes.  NOTE that if an
%% \altaffiliation command is used it must come BEFORE the \affiliation call,
%% right after the \author command, in order to place the footnotes in
%% the proper location.
%%
%% Use \email to set provide email addresses. Each \email will appear on its
%% own line so you can put multiple email address in one \email call. A new
%% \correspondingauthor command is available in V6.3 to identify the
%% corresponding author of the manuscript. It is the author's responsibility
%% to make sure this name is also in the author list.
%%
%% While authors can be grouped inside the same \author and \affiliation
%% commands it is better to have a single author for each. This allows for
%% one to exploit all the new benefits and should make book-keeping easier.
%%
%% If done correctly the peer review system will be able to
%% automatically put the author and affiliation information from the manuscript
%% and save the corresponding author the trouble of entering it by hand.

\correspondingauthor{A.~A.~Miller}
\email{amiller@northwestern.edu}


%% Note that the \and command from previous versions of AASTeX is now
%% depreciated in this version as it is no longer necessary. AASTeX 
%% automatically takes care of all commas and "and"s between authors names.

%% AASTeX 6.3 has the new \collaboration and \nocollaboration commands to
%% provide the collaboration status of a group of authors. These commands 
%% can be used either before or after the list of corresponding authors. The
%% argument for \collaboration is the collaboration identifier. Authors are
%% encouraged to surround collaboration identifiers with ()s. The 
%% \nocollaboration command takes no argument and exists to indicate that
%% the nearby authors are not part of surrounding collaborations.

%% Mark off the abstract in the ``abstract'' environment. 
\begin{abstract}


\end{abstract}

%% Keywords should appear after the \end{abstract} command. 
%% See the online documentation for the full list of available subject
%% keywords and the rules for their use.
\keywords{}

%% From the front matter, we move on to the body of the paper.
%% Sections are demarcated by \section and \subsection, respectively.
%% Observe the use of the LaTeX \label
%% command after the \subsection to give a symbolic KEY to the
%% subsection for cross-referencing in a \ref command.
%% You can use LaTeX's \ref and \label commands to keep track of
%% cross-references to sections, equations, tables, and figures.
%% That way, if you change the order of any elements, LaTeX will
%% automatically renumber them.
%%
%% We recommend that authors also use the natbib \citep
%% and \citet commands to identify citations.  The citations are
%% tied to the reference list via symbolic KEYs. The KEY corresponds
%% to the KEY in the \bibitem in the reference list below. 

\section{Introduction} \label{sec:intro}

The proliferation of wide-field time-domain surveys over the past $\sim$decade
has led to the discovery of a bevy of novel extragalactic transients
\citep[e.g.,][]{quimby11,Gezari12,Gal-Yam14,Abbott17a,Prentice18,
IceCube-Collaboration18}. While these wide-field surveys have been enabled by
significant advances in detector technology, software has proven equally
important \citep[e.g.,][]{Masci17,Masci19,Smith20,Jones20} as many of these
critical discoveries have been facilitiated by the rapid identification and
dissemination of new transient candidates in near real time
\citep[e.g.,][]{Patterson19}.

Reliable catalogs identifying stars and galaxies, or similarly unresolved and
resolved sources, are an essential cog in the machinery necessary to identify
extragalactic transients. On a nightly basis, time-domain sureys are inundated
with transient candidates, the vast majority of which are considered ``bogus''
\citep{Bloom12}. Despite sophisticated software capable of whittling down the
number of likely transients by several orders of magnitude
\citep[e.g.,][]{Brink13,Goldstein15,Duev19,Smith20}, the number of candidates
still vastly outpaces the spectroscopic resources necessary to classify
everything that varies \citep[e.g.,][]{Kulkarni20}. The aforementioned
star--galaxy catalogs therefore play an essential role in the search for
extragalactic transients by removing candidates that are associated with
stellar-like objects that are therefore likely to be Galactic in origin.

To this end we previously created the PanSTARRS-1 Point Source Catalog
\citep[PS1 PSC; Tachibana \& Miller 2018, hereafter ][]{Tachibana18}, which
provides probabilistic point-source like classifications for $\sim$1.5 billion
sources detected by PanSTARRS-1 \citep[PS1;][]{Chambers16}. This catalog has
been deployed by the Zwicky Transient Facility \citep[ZTF;][]{Bellm19} and
other surveys \citep{Smith20,Moller20} to identify likely extragalactic
transients. The PS1 PSC has been demonstrated to be an important ingredient in
the systematic search for extragalactic transients \citep{Fremling20,De20}.

\todo{??? Add quick explanation of mean, forced, and stack phot from PS1 ???}

A downside to the PS1 PSC is that it does not provide classifications for
sources that are not ``detected'' in the PS1 \texttt{StackObjectThin} table
\citep[the definition of a PS1 stack detection is given in][]{Tachibana18}. Of
the $\sim$3.5 billion sources in the PS1 \texttt{stackObjectThin} table, the
vast majority of the $\sim$2 billion missing from the PS1 PSC are either
spurious or have an extremely low signal-to-noise ratio (S/N), such that the
methods in \citet{Tachibana18} would not provide a reliable classification.
There are an additional \todo{$\sim$140; need final number} million high-S/N
sources missing from the PS1 PSC because they have multiple entries in the PS1
\texttt{StackObjectThin} table with $\mathtt{primaryDetection} =
1$.\footnote{The \texttt{primaryDetection} flag identifies which entry within
the PS1 \texttt{stackObjectThin} table was detected from the stack image that
provides the best coverage with minimal projection stretching. Roughly 0.5\%
of the \texttt{stackObjectThin} sources have multiple rows with
$\mathtt{primaryDetection} = 1$, effectively making it impossible to identify
which row should be used in conjunction with the model from
\citet{Tachibana18}.} These sources, many of which are moderately bright
stars, have an $\mathtt{sg_score} = 0.5$ within the ZTF alert packets, i.e.,
there classification is ambiguous, and as a result stars can end up in
extragalactic transient streams.

Here, we present an update to the PS1 PSC by classifying these
\todo{$\sim$140; need final number} million sources using photometry from the
\texttt{ForcedObjectMean} table in the PS1 database. While we previously
showed that the PS1 forced photometry does not separate resolved and
unresolved sources with the same fidelity as its stack photometry
\citep{Tachibana18}, we can, using a nearly identical methodology as
\citet{Tachibana18}, nevertheless achieve similar performance with the PS1
forced photometry. We apply our new model to the \todo{$\sim$140; need final
number} million previously unclassified sources, providing a new and useful
supplement to the PS1 PSC.\footnote{During the preparation of this manuscript
\citet{Beck20} published a new machine learning catalog (PS1-STRM) to separate
the $\sim$3.5 billion sources in the PS1 \texttt{ForcedMeanObject} table. The
main difference between the PS1 PSC and PS1-STRM is the training set used by
each machine learning model, which we discuss in more detail below.}

\acknowledgments

\vspace{5mm}
\facilities{}

%% Similar to \facility{}, there is the optional \software command to allow 
%% authors a place to specify which programs were used during the creation of 
%% the manuscript. Authors should list each code and include either a
%% citation or url to the code inside ()s when available.

\software{}

\bibliography{/Users/adamamiller/Documents/tex_stuff/papers}{}
\bibliographystyle{aas_arxiv}

%% Include this line if you are using the \added, \replaced, \deleted
%% commands to see a summary list of all changes at the end of the article.
%\listofchanges

\end{document}

% End of file `sample63.tex'.
